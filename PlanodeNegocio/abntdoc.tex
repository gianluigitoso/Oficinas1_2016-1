\documentclass[draft]{article}
%\documentclass[a4paper,12pt]{article}
\usepackage{verbatim}
\usepackage[brazil]{babel}
\usepackage[utf8]{inputenc}

%\newcommand{\identacao}{\\[1mm] \hspace*{\parindent}}
\newcommand{\identacao}{\\[1mm] \hspace*{3cm}}
\hyphenation{pri-mei-ros}

\begin{document}

\title{Controle e Manutenção de Hortas e Jardins de Baixo Custo com Arduino}
\author{Gianluigi Dal Toso, Guilmour Rossi, Leandro Vieira, Luís Felipe\\Werlang, Mateus Gomes e Samuel Henrique\\\texttt{contato@guilmour.com}\\\\\\
\multicolumn{1}{p{.95\textwidth}}{\centering\emph{Engenharia de Computação - Universidade Tecnológica Federal do Paraná (UTFPR) \\Oficina de Integração I}}}

\date{Junho de 2016}

\maketitle

\newpage

\tableofcontents

\newpage

\section{Resumo}
  O projeto consiste no desenvolvimento e construção de um sistema de automação residencial com \textit{Arduino} para a abertura e fechamento de portas, portões, trancas ou meios de acesso por meio de dispositivos digitais, tais como: celulares \textit{smartphones}, computadores ou dispositivos similares com acesso web; substituindo assim o uso de chaves físicas convencionais, trazendo um aumento de segurança com o monitoramento remoto sobre o estado da porta (aberta ou fechada) e quem teve acesso por ela.
  \textbf{\\Palavras-chave:} tranca-eletrônica, tranca-eletromecânica, arduino, android, monitoramento, segurança, automação residencial.

\newpage

\section{Motivação}
\subsection{Uma Breve História da Domótica}
O auxílio de máquinas e métodos automáticos para ajudar o ser humano em seu dia-a-dia é algo que vem sendo buscado à muito tempo, sendo como um desejo inerente ao ser humano na tentativa de se obter progresso e melhores condições de vida. Depois da Revolução Industrial no final do século XVII, isso se intensificou, deixando claro a força e a importância da tecnologia e das máquinas automatizadas, e que estas agora fariam parte de nossa sociedade - aquela época apenas para fins fabris \cite{hobsbawm1} -. Muitos anos de passaram e já na década de 1920 as máquinas automatizadas começaram a ter um novo destino: a casa de famílias comuns. Onde os fabricantes já usavam de uma visão otimista e de termos como \textit{casa do futuro} \cite{bolzani}, a fim de promover seus produtos e levar ao consumidor uma imagem de progresso e de que o eletrodoméstico levaria um aumento de bem-estar para a sociedade. Assim, junto até das histórias de ficções-científicas mais criativas, foram sendo criados cada vez mais equipamentos e máquinas de inúmeros tipos e utilidades, tentando fazê-lo não apenas ter praticidade ou alguma forma de lazer, mas também economizar tempo nas tarefas do dia-a-dia de forma adequada e reduzir ao máximos seus esforços, onde desafios da domótica estão cada vez mais atrelados as tarefas mais comuns e de baixo nível de uma residência.

\subsection{O que é Domótica}
Domótica é a tecnologia responsável pela administração de todos os recursos de qualquer tipo de habitação. O termo vem da junção das palavras “Domus”, que significa casa, e “Robótica”, que claro, está ligada ao ato de automatizar, de realizar ações de forma automática. \cite{chamusca}


Tais progressos podem ser percebidos em muitas áreas, mas muitas vezes por serem tecnologias caras ou avançadas, acabam não sendo difundidas em larga escala para todos. Este é o caso da automação industrial, de portas automáticas, trancas eletrônicas e similares. Cuja implementação acaba sendo feita apenas em locais que protegem objetos de valor, em grandes construções comerciais ou em residências das classes com maior poder aquisitivo, deixando a outra parcela da sociedade sem os benefícios práticos e de segurança destas tecnologias.

Segurança esta, que é um fator essencial em qualquer ambiente pré-disposto a ser corrompido, como bancos, lojas, casas e apartamentos que estão vulneráveis tanto fisicamente,  por meio de portas, portões ou grades, quanto logicamente, por sistemas que podem ou não utilizar a internet. A maioria dos lares brasileiros ainda estão inseguros por sistemas de “controle de pessoas” ultrapassados, que são facilmente corrompidos, colocando vidas em risco, assim como a perda de um bem material. Dessa forma, aliando sistemas digitas com sistemas físicos poderemos ter um avanço enorme na segurança das residências ou estabelecimentos. Para exemplificar, pensaremos na porta de nossa casa: o que utilizamos para abri-lá e fecha-lá? Existe chave e maçaneta? Caso esquecermos de fecha-lá será nos avisado? O que acontece caso alguém tente utilizar outra chave? E se perdermos a nossa chave?

Assim, é necessário um sistema que resolva essas questões de insegurança que muitos ainda convivem aliando-se as tecnologias atuais. Um sistema inteligente que não mais utilize uma mera chave, que é uma ferramenta facilmente copiável, mas sim um celular, ou cartão, ou algum \textit{gadget} que possa ser utilizado, elevando o nível de confiabilidade e desinformação das pessoas, proporcionando uma vida mais tecnológica e segura.

Perder a chave não será mais um problema, pois sempre estaremos com com um mini-cartão ou alguma pulseira que contem uma ou mais  senhas criptografadas para dar acesso a diferentes portões. Também será possível utilizar um celular com \textit{NFC} ou \textit{Bluetooth} como “chave” de acesso  como segunda opção, assim como, saber exatamente se a porta esta aberta ou fechada, liberar um “chave” por algum tempo pre estabelecido para um funcionário. A ideia seria  que esse funcionário pudesse baixar o aplicativo e digitar a senha de acesso, que será feita avaliada através do celular com o sistema.

Definição do que pode ser feito:
\begin{itemize}
\item Abrir e fechar uma porta, portão ou meio de acesso por meio de dispositivos digitais, tais como: celulares, cartões ou pulseiras.
\item Liberar uma “Chave” por um tempo determinado.
\item Saber se a porta esta aberta ou fechada e efetuar sua troca remotamente, assim como saber quem teve acesso.

\end{itemize}

%\appendix
%\addcontentsline{toc}{chapter}{REFER\^ENCIAS}
Feyman \cite{Feynman}
%\bibliographystyle{unsrt}
\bibliographystyle{abnt}
\bibliography{abntdoc}
%\bibliography{periodicos_extenso,abntdoc}
%\bibliography{macros,abntdoc}

\end{document}
