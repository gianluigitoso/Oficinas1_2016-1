\documentclass[12pt]{article}

\usepackage{sbc-template}
\usepackage{multicol}

\usepackage{graphicx,url}

\usepackage[brazil]{babel}   
%\usepackage[latin1]{inputenc}  
\usepackage[utf8]{inputenc}  


     
\sloppy

\title{Controle e Manutenção de Hortas de Baixo Custo com Arduino}

\author{Gianluigi Dal Toso, Guilmour Rossi, Leandro Vieira, Luís Felipe Werlang, \\Mateus Gomes e Samuel Henrique}


\address{Curso de Engenharia de Computação -- Universidade Tecnológica Federal do Paraná 
\\(UTFPR) - Oficinas de Integração I}
  



\begin{document} 


\maketitle
\begin{center}
\texttt{contato@guilmour.com}
\end{center}

%
%\begin{abstract}
%  This meta-paper describes the style to be used in articles and short papers
 % for SBC conferences. For papers in English, you should add just an abstract
  %while for the papers in Portuguese, we also ask for an abstract in
  %Portuguese (``resumo''). In both cases, abstracts should not have more than
  %10 lines and must be in the first page of the paper.
%\end{abstract}
     
\begin{resumo} 
  Este relatório tem como objetivo relatar os detalhes técnicos e práticos do desenvolvimento e construção de um sistema de controle e manutenção para jardins e hortas agroecológicas e de baixo custo em quintais e apartamentos, usando a plataforma de prototipagem eletrônica, Arduino. Com o uso de sensores de umidade, temperatura, luz, e o correto manuseio dessas informações, pode-se encontrar as condições ideias para cultivo nestes locais, maximizando o crescimento e qualidade das plantas e garantindo um cultivo saudável mesmo em lares onde a manutenção de uma pequena horta não se encaixaria na rotina dos moradores.
  \textbf{\\\\Palavras-chave:} horta inteligente, arduino, manutenção de jardim, cultivo saudável, agroecologia.
\end{resumo}




\section{Apresentação}











%\section{Sections and Paragraphs}


%\subsection{Subsections}



%\section{Figures and Captions}\label{sec:figs}





%\section{Images}



%\section{References}



\bibliographystyle{apalike}
\bibliography{sbc-template}

\end{document}
