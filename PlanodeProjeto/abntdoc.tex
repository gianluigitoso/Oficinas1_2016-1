\documentclass[a4paper,12pt]{article}
\usepackage{verbatim}
\usepackage[brazil]{babel}
\usepackage[utf8]{inputenc}
\usepackage{nomencl}
\usepackage{indentfirst}


%\newcommand{\identacao}{\\[1mm] \hspace*{\parindent}}
\newcommand{\identacao}{\\[1mm] \hspace*{3cm}}
\hyphenation{pri-mei-ros}


\begin{document}

\title{Controle e Manutenção de Hortas e Jardins de Baixo Custo com Arduino}
\author{Gianluigi Dal Toso, Guilmour Rossi, Leandro Vieira, Luís Felipe\\Werlang, Mateus Gomes e Samuel Henrique\\\texttt{}\\\\\\
\multicolumn{1}{p{.95\textwidth}}{\centering\emph{Engenharia de Computação - Universidade Tecnológica Federal do Paraná (UTFPR) \\Oficina de Integração I}}}

\date{Curitiba, \\Junho de 2016}

\maketitle

\newpage


\tableofcontents


\newpage

\listoffigures

\newpage

\listoftables

\makenomenclature

\newpage
\section{Introdução}

	O modelo atual de agricultura adotado nas últimas décadas, apesar de sua elevada efetividade de produção, tem se mostrado destrutivo ao planeta, com inúmeros impactos ambientais e também sociais; erosões do solo, contaminação de rios, redução da biodiversidade, exclusão social e etc. \cite[p.~23]{medeiros}. A fim de minimizar ou até eliminar estes problemas, novas alternativas de agricultura vem sendo buscadas e aperfeiçoadas

Neste ponto  de um sistema de controle e manutenção para jardins e hortas agroecológicas e de baixo custo em quintais e apartamentos, usando a plataforma de prototipagem eletrônica, Arduino. Com o uso de sensores de umidade, temperatura, luz, e o correto manuseio dessas informações, pode-se encontrar as condições ideais para cultivo nestes locais, maximizando o crescimento e qualidade das plantas e garantindo um cultivo saudável mesmo em lares onde a manutenção de uma pequena horta não se encaixaria na rotina dos moradores.
  \textbf{\\\\Palavras-chave:} tranca-eletrônica, tranca-eletromecânica, arduino, monitoramento, segurança, automação residencial.

\subsection{Objetivo Geral}
Indicar o objetivo geral.

\subsection{Objetivos Específicos}
Indicar cada objetivo específico do projeto.

\newpage

\section{Metodologia}
É uma descrição técnica de como será desenvolvido ou foi desenvolvido o trabalho. Devem estar detalhadas, de forma lógica, linear e cronológica, todas as etapas do projeto.
Deve ser explicado como o produto será gerado, quais são os principais fundamentos (algoritmos, paradigmas, teorias) e tecnologias (ambientes de desenvolvimento, linguagens de programação, plataformas de hardware) a serem empregados.
\subsection{Fundamentos}
Indicar quais são os quais são os principais fundamentos (algoritmos, paradigmas, teorias) a serem empregados.
Cada fundamento utilizado deve ser justificado.
\subsection{Tecnologias}
Indicar quais são os quais são as principais tecnologias (ambientes de desenvolvimento, linguagens de programação, plataformas de hardware) a serem empregadas.
Cada tecnologia utilizada deve ser justificada.

\section{Recursos De Hardware E Software}
Apresente aqui todos os recursos essenciais ao desenvolvimento do projeto. O(s) aluno(s) deve(m) mencionar também a origem dos recursos (próprios, externos ou da UTFPR) e a viabilidade do projeto.

\subsection{Recursos de Hardware}
Indique aqui os recursos de hardware como componentes digitais, analógicos, fontes de alimentação, baterias, sensores, atuadores, entre outros. Especifique a origem dos recursos de hardware.

\subsection{Recursos de Software}
Devem ser apresentados aqui os recursos de software, incluindo os principais fundamentos (teorias, algoritmos, paradigmas) e tecnologias (ambientes de desenvolvimento, linguagens de programação) a serem empregados. Especifique a origem dos recursos de software.

\subsection{Viabilidade}
Indique aqui a viabilidade técnica e financeira do projeto.

\section{Contexto}
Neste item deve ser explicitado quem é o cliente do projeto, se o projeto faz parte de um projeto maior, se ele é contratado por uma empresa externa a UTFPR e que se responsabilizará pelos recursos, entre outras possibilidades.

\section{Projeto a ser Desenvolvido e Resultados Iniciais do Mesmo}
Nesta seção deverá ser feita uma descrição funcional da solução proposta. O projeto deverá se decomposto em módulos funcionais (incluindo software e hardware).
Deverá ser apresentado um diagrama em blocos mostrando uma visão geral do sistema a ser desenvolvido, isto é, como os módulos funcionais se relacionam.
Cada módulo funcional deve ser descrito individualmente.
Deve ser indicado também, de maneira sucinta, como cada módulo será implementado (linguagem de programação, hardware, etc).

\subsection{Projeto de Software}
O projeto de software do sistema deve ser feito através da Análise Orientada a Objetos, sendo necessária a apresentação dos seguintes itens:
Levantamento de Requisitos:
Requisitos Funcionais: descrevem o comportamento do sistema, suas ações para cada entrada, ou seja, é aquilo que tem que ser feito pelo sistema.
Requisitos Não-Funcionais: são aqueles que expressam como deve ser feito. Em geral, se relacionam com padrões de qualidade como confiabilidade, performance, robustez, entre outros. São muito importantes, pois definem se o sistema será eficiente para a tarefa que se propõe a fazer ou não. Um sistema ineficiente certamente não será utilizado. Neles também são apresentados restrições e especificações de uso para os requisitos funcionais.
Diagramas de Caso de Uso com:
Descrição: o que o diagrama de casos de uso representa.
Ator Principal: tem objetivos de usuários satisfeitos por uso do sistema a ser desenvolvido.
Ator de Suporte: fornece um serviço (como, por exemplo, informações) para o sistema. Apresentar, caso o mesmo exista.
Ator de Bastidor: tem interesse no comportamento do caso de uso, mas não é um ator principal ou de suporte (como, por exemplo, um órgão governamental). Apresentar, caso o mesmo exista.
Pré-condições: condições necessárias para o diagrama poder ser executado.
Pós-condições: resultados da execução do diagrama de casos de uso.
Fluxo Básico: tarefas que consideram situações de perfeição.
Fluxo Alternativo: variações em relação ao fluxo básico de eventos.
Regras de Negócio: condições ou restrições sobre os processos de negócio.
Diagrama de Classes, juntamente com:
Dicionário de Informações.
Diagrama de Objetos (Instâncias).
Diagramas de Seqüência para cada Caso de Uso.

\subsection{Projeto de Hardware}
Deverá ser apresentado um diagrama em blocos mostrando uma visão geral do sistema a ser desenvolvido, isto é, como os módulos funcionais se relacionam.
Cada módulo funcional deve ser descrito individualmente, utilizando os conceitos desenvolvidos nas disciplinas de Arquitetura e Organização de Computadores e Sistemas Microcontrolados.

\section{Procedimentos de Teste e Validação}
Aqui deve ser indicado como os diversos módulos do projeto serão testados e validados individualmente e em conjunto.
Deverão ser estabelecidos os critérios de aceitação do projeto em termos de desempenho e do cumprimento de seus aspectos funcionais.
Descrever aqui os testes de caixa preta para cada Caso de Uso do sistema.

\section{Análise de riscos}
Análise dos riscos potenciais do projeto e do impacto desses problemas no sucesso ou fracasso do mesmo. Deve ser seguida a metodologia apresentada na Disciplina de Engenharia de Software.


\section{Cronograma do Projeto}
Descrição de todas as fases do projeto, incluindo a seqüência de desenvolvimento de cada módulo, o tempo de desenvolvimento e as fases de teste e documentação. Para isto, devem ser utilizadas as seguintes ferramentas:
Gráfico de Gant.
Rede PERT/CPM.

\section{Conclusões}
As conclusões do trabalho devem ser expostas de maneira clara, lógica e concisa. Fale das dificuldades e facilidades encontradas durante a execução da proposta até o momento.

%\appendix
%\addcontentsline{toc}{chapter}{REFER\^ENCIAS}
Feyman asdasdasd \cite{altieri1}
%\bibliographystyle{unsrt}
\renewcommand\refname{}
\bibliographystyle{abnt}

\newpage
\section{Referências Bibliográficas}
\renewcommand\refname{}
\bibliography{abntdoc}
%\bibliography{periodicos_extenso,abntdoc}
%\bibliography{macros,abntdoc}



\end{document}
